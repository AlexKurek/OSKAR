%
% This is an autogenerated file - do not edit!
%


\fontsize{8}{10}\selectfont
\begin{center}
\begin{longtable}{|L{9cm}|L{8.5cm}|L{4.0cm}|L{1.7cm}|}

\hline
  \rowcolor{lightgray}
  {\textbf{Key}} &
  {\textbf{Description}} &
  {\textbf{Allowed values}} &
  {\textbf{Default}} \\ \hline
\endfirsthead
\hline
  \rowcolor{lightgray}
  {\textbf{Key}} &
  {\textbf{Description}} &
  {\textbf{Allowed values}} &
  {\textbf{Default}} \\ \hline
\endhead
  \multicolumn{4}{l}{{Continued on next page\ldots}} \\
\endfoot
 
\endlastfoot
input\_directory\textsuperscript{\textbf{\dag}}
&
{Path to a directory containing the telescope configuration data. See the accompanying documentation for a description of an OSKAR telescope model directory.}
&
Path name
&

\\
\hline

longitude\_deg
&
{Telescope centre (east) longitude, in degrees.}
&
Double
&
0.0
\\
\hline

latitude\_deg
&
{Telescope centre latitude, in degrees.}
&
Double
&
0.0
\\
\hline

altitude\_m
&
{Telescope centre altitude, in metres.}
&
Double
&
0.0
\\
\hline

station\_type
&
{The type of each station in the interferometer. A simple, time-invariant Gaussian station beam can be used instead of an aperture array beam if required for testing. All station beam effects can be disabled by selecting `Isotropic beam'.}
&
{One of the following:}
{\begin{itemize}[leftmargin=5ex, topsep=0pt, partopsep=0pt, itemsep=2pt, parsep=0pt]
\vspace{4pt}\item {Aperture array}
\item {Isotropic beam}
\item {Gaussian beam}
\item {VLA (PBCOR)}
\end{itemize}
}
&
Aperture
\\
\hline

normalise\_beams\_at\_phase\_centre
&
{If \textbf{true}, then scale the amplitude of every station beam at the interferometer phase centre to precisely 1.0 for each time snapshot. This effectively performs an amplitude calibration for a source at the phase centre.}
&
Bool
&
true
\\
\hline

pol\_mode
&
{The polarisation mode of simulations which use the telescope model. If this is \textbf{Scalar}, then only Stokes I visibility data will be simulated, and scalar element responses will be used when evaluating station beams. If this is \textbf{Full} (the default) then correlation products from both polarisations will be simulated. \textbf{Note that scalar mode can be significantly faster.}}
&
{One of the following:}
{\begin{itemize}[leftmargin=5ex, topsep=0pt, partopsep=0pt, itemsep=2pt, parsep=0pt]
\vspace{4pt}\item {Full}
\item {Scalar}
\end{itemize}
}
&
Full
\\
\hline

allow\_station\_beam\_duplication
&
{If enabled, and if all stations are identical, all station beam responses will be copied from the first. This can reduce the simulation time, but \textbf{when using a telescope model with long baselines, source positions will not shift with respect to each station's horizon if this option is enabled.} This setting has no effect if all stations are not identical.}
&
Bool
&
false
\\
\hline

aperture\_array/array\_pattern/enable
&
{If true, then the contribution to the station beam from the array pattern (given by beamforming the antennas in the station) is evaluated. If false, then the array pattern is ignored.}
&
Bool
&
true
\\
\hline

aperture\_array/array\_pattern/normalise
&
{If true, the amplitude of each station beam will be divided by the number of antennas in the station; if false, then this normalisation is not performed. Note, however, that global beam normalisation is still possible.}
&
Bool
&
false
\\
\hline

aperture\_array/array\_pattern/element/gain
&
{Mean element amplitude gain factor. If set (and $>$ 0.0), this will override the contents of the station files.}
&
Double
&
0.0
\\
\hline

aperture\_array/array\_pattern/element/gain\_error\_fixed
&
{Systematic element amplitude gain standard deviation. If set, this will override the contents of the station files.}
&
Double
&
0.0
\\
\hline

aperture\_array/array\_pattern/element/gain\_error\_time
&
{Time-variable element amplitude gain standard deviation. If set, this will override the contents of the station files.}
&
Double
&
0.0
\\
\hline

aperture\_array/array\_pattern/element/phase\_error\_fixed\_deg
&
{Systematic element phase standard deviation. If set, this will override the contents of the station files.}
&
Double
&
0.0
\\
\hline

aperture\_array/array\_pattern/element/phase\_error\_time\_deg
&
{Time-variable element phase standard deviation. If set, this will override the contents of the station files.}
&
Double
&
0.0
\\
\hline

aperture\_array/array\_pattern/element/position\_error\_xy\_m
&
{The standard deviation of the antenna xy-position uncertainties. If set, this will override the contents of the station files.}
&
Double
&
0.0
\\
\hline

aperture\_array/array\_pattern/element/x\_orientation\_error\_deg
&
{The standard deviation of the antenna X-dipole orientation error. If set, this will override the contents of the station files.}
&
Double
&
0.0
\\
\hline

aperture\_array/array\_pattern/element/y\_orientation\_error\_deg
&
{The standard deviation of the antenna Y-dipole orientation error. If set, this will override the contents of the station files.}
&
Double
&
0.0
\\
\hline

aperture\_array/array\_pattern/element/seed\_gain\_errors
&
{Random number generator seed used for systematic gain error distribution.}
&
Integer $\geq$ 1, or `time'
&
1
\\
\hline

aperture\_array/array\_pattern/element/seed\_phase\_errors
&
{Random number generator seed used for systematic phase error distribution.}
&
Integer $\geq$ 1, or `time'
&
1
\\
\hline

aperture\_array/array\_pattern/element/seed\_time\_variable\_errors
&
{Random number generator seed used for time variable error distributions.}
&
Integer $\geq$ 1, or `time'
&
1
\\
\hline

aperture\_array/array\_pattern/element/seed\_position\_xy\_errors
&
{Random number generator seed used for antenna xy-position error distribution.}
&
Integer $\geq$ 1, or `time'
&
1
\\
\hline

aperture\_array/array\_pattern/element/seed\_x\_orientation\_error
&
{Random number generator seed used for antenna X dipole orientation error distribution.}
&
Integer $\geq$ 1, or `time'
&
1
\\
\hline

aperture\_array/array\_pattern/element/seed\_y\_orientation\_error
&
{Random number generator seed used for antenna Y dipole orientation error distribution.}
&
Integer $\geq$ 1, or `time'
&
1
\\
\hline

aperture\_array/element\_pattern/functional\_type
&
{The type of functional pattern to apply to the elements, if not using a numerically-defined pattern.}
&
{One of the following:}
{\begin{itemize}[leftmargin=5ex, topsep=0pt, partopsep=0pt, itemsep=2pt, parsep=0pt]
\vspace{4pt}\item {Dipole}
\item {Geometric dipole}
\item {Isotropic (unpolarised)}
\end{itemize}
}
&
Dipole
\\
\hline

aperture\_array/element\_pattern/dipole\_length
&
{The length of the dipole, if using dipole elements.}
&
Double
&
0.5
\\
\hline

aperture\_array/element\_pattern/dipole\_length\_units
&
{The units used to specify the dipole length (metres or wavelengths), if using dipole elements.}
&
{One of the following:}
{\begin{itemize}[leftmargin=5ex, topsep=0pt, partopsep=0pt, itemsep=2pt, parsep=0pt]
\vspace{4pt}\item {Wavelengths}
\item {Metres}
\end{itemize}
}
&
Wavelengths
\\
\hline

aperture\_array/element\_pattern/enable\_numerical
&
{If \textbf{true}, make use of any available numerical element pattern files. If numerical pattern data are missing, the functional type will be used instead.}
&
Bool
&
true
\\
\hline

aperture\_array/element\_pattern/taper/type
&
{The type of tapering function to apply to the element pattern.}
&
{One of the following:}
{\begin{itemize}[leftmargin=5ex, topsep=0pt, partopsep=0pt, itemsep=2pt, parsep=0pt]
\vspace{4pt}\item {None}
\item {Cosine}
\item {Gaussian}
\end{itemize}
}
&
None
\\
\hline

aperture\_array/element\_pattern/taper/cosine\_power
&
{If a cosine element taper is selected, this setting gives the power of the cosine(theta) function.}
&
Double
&
1.0
\\
\hline

aperture\_array/element\_pattern/taper/gaussian\_fwhm\_deg
&
{If a Gaussian element taper is selected, this setting gives the full-width half maximum value of the Gaussian, in degrees.}
&
Double
&
45.0
\\
\hline

gaussian\_beam/fwhm\_deg
&
{For stations using a simple Gaussian beam, this setting gives the full-width half maximum value of the Gaussian station beam at the reference frequency, in degrees.}
&
Double
&
1.0
\\
\hline

gaussian\_beam/ref\_freq\_hz
&
{The reference frequency of the specified FWHM, in Hz.}
&
Double
&
0.0
\\
\hline

output\_directory
&
{Path used to save the final telescope model directory, excluding any element pattern data (useful for debugging). Leave blank if not required.}
&
Path name
&

\\
\hline

\end{longtable}
\end{center}
\normalsize
\newpage
