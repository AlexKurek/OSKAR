%
% This is an autogenerated file - do not edit!
%


\fontsize{8}{10}\selectfont
\begin{center}
\begin{longtable}{|L{0.4cm}|L{9cm}|p{8.5cm}|L{3.0cm}|L{1.7cm}|}

\hline
  \rowcolor{lightgray}
  {\textbf{ID}} &
  {\textbf{Key}} &
  {\textbf{Description}} &
  {\textbf{Allowed values}} &
  {\textbf{Default}} \\[0.5ex] \hline
\endfirsthead
\hline
  \rowcolor{lightgray}
  {\textbf{ID}} &
  {\textbf{Key}} &
  {\textbf{Description}} &
  {\textbf{Allowed values}} &
  {\textbf{Default}} \\[0.5ex] \hline
\endhead
  \multicolumn{5}{l}{{Continued on Next Page\ldots}} \\
\endfoot
  
\endlastfoot
102
&
input\_directory\textsuperscript{\textbf{\dag}}
&
Path to a directory containing the telescope configuration data. See the accompanying documentation for a description of an OSKAR telescope model directory.
&
Allowed values
&

\\
\hline

103
&
longitude\_deg
&
Telescope centre (east) longitude, in degrees.
&
Double
&
0.0
\\
\hline

104
&
latitude\_deg
&
Telescope centre latitude, in degrees.
&
Double
&
0.0
\\
\hline

105
&
altitude\_m
&
Telescope centre altitude, in metres.
&
Double
&
0.0
\\
\hline

106
&
station\_type
&
The type of each station in the interferometer. A simple, time-invariant Gaussian station beam can be used instead of an aperture array beam if required for testing. All station beam effects can be disabled by selecting `Isotropic beam'.
&
Allowed values
&
Aperture
\\
\hline

107
&
normalise\_beams\_at\_phase\_centre
&
If \textbf{true}, then scale the amplitude of every station beam at the interferometer phase centre to precisely 1.0 for each time snapshot. This effectively performs an amplitude calibration for a source at the phase centre.
&
Bool
&
true
\\
\hline

108
&
pol\_mode
&
The polarisation mode of simulations which use the telescope model. If this is \textbf{Scalar}, then only Stokes I visibility data will be simulated, and scalar element responses will be used when evaluating station beams. If this is \textbf{Full} (the default) then correlation products from both polarisations will be simulated. \textbf{Note that scalar mode can be significantly faster.}
&
Allowed values
&
Full
\\
\hline

109
&
aperture\_array/array\_pattern/enable
&
If true, then the contribution to the station beam from the array pattern (given by beamforming the antennas in the station) is evaluated. If false, then the array pattern is ignored.
&
Bool
&
true
\\
\hline

110
&
aperture\_array/array\_pattern/normalise
&
If true, the amplitude of each station beam will be divided by the number of antennas in the station; if false, then this normalisation is not performed. Note, however, that global beam normalisation is still possible.
&
Bool
&
false
\\
\hline

111
&
aperture\_array/array\_pattern/element/gain
&
Mean element amplitude gain factor. If set (and $>$ 0.0), this will override the contents of the station files.
&
Double
&
0.0
\\
\hline

112
&
aperture\_array/array\_pattern/element/gain\_error\_fixed
&
Systematic element amplitude gain standard deviation. If set, this will override the contents of the station files.
&
Double
&
0.0
\\
\hline

113
&
aperture\_array/array\_pattern/element/gain\_error\_time
&
Time-variable element amplitude gain standard deviation. If set, this will override the contents of the station files.
&
Double
&
0.0
\\
\hline

114
&
aperture\_array/array\_pattern/element/phase\_error\_fixed\_deg
&
Systematic element phase standard deviation. If set, this will override the contents of the station files.
&
Double
&
0.0
\\
\hline

115
&
aperture\_array/array\_pattern/element/phase\_error\_time\_deg
&
Time-variable element phase standard deviation. If set, this will override the contents of the station files.
&
Double
&
0.0
\\
\hline

116
&
aperture\_array/array\_pattern/element/position\_error\_xy\_m
&
The standard deviation of the antenna xy-position uncertainties. If set, this will override the contents of the station files.
&
Double
&
0.0
\\
\hline

117
&
aperture\_array/array\_pattern/element/x\_orientation\_error\_deg
&
The standard deviation of the antenna X-dipole orientation error. If set, this will override the contents of the station files.
&
Double
&
0.0
\\
\hline

118
&
aperture\_array/array\_pattern/element/y\_orientation\_error\_deg
&
The standard deviation of the antenna Y-dipole orientation error. If set, this will override the contents of the station files.
&
Double
&
0.0
\\
\hline

119
&
aperture\_array/array\_pattern/element/seed\_gain\_errors
&
Random number generator seed used for systematic gain error distribution.
&
`time' or integer seed $\geq$ 1
&
1
\\
\hline

120
&
aperture\_array/array\_pattern/element/seed\_phase\_errors
&
Random number generator seed used for systematic phase error distribution.
&
`time' or integer seed $\geq$ 1
&
1
\\
\hline

121
&
aperture\_array/array\_pattern/element/seed\_time\_variable\_errors
&
Random number generator seed used for time variable error distributions.
&
`time' or integer seed $\geq$ 1
&
1
\\
\hline

122
&
aperture\_array/array\_pattern/element/seed\_position\_xy\_errors
&
Random number generator seed used for antenna xy-position error distribution.
&
`time' or integer seed $\geq$ 1
&
1
\\
\hline

123
&
aperture\_array/array\_pattern/element/seed\_x\_orientation\_error
&
Random number generator seed used for antenna X dipole orientation error distribution.
&
`time' or integer seed $\geq$ 1
&
1
\\
\hline

124
&
aperture\_array/array\_pattern/element/seed\_y\_orientation\_error
&
Random number generator seed used for antenna Y dipole orientation error distribution.
&
`time' or integer seed $\geq$ 1
&
1
\\
\hline

125
&
aperture\_array/element\_pattern/functional\_type
&
The type of functional pattern to apply to the elements, if not using a numerically-defined pattern.
&
Allowed values
&
Dipple
\\
\hline

126
&
aperture\_array/element\_pattern/dipole\_length
&
The length of the dipole, if using dipole elements.
&
Double
&
0.5
\\
\hline

127
&
aperture\_array/element\_pattern/dipole\_length\_units
&
The units used to specify the dipole length (metres or wavelengths), if using dipole elements.
&
Allowed values
&
Wavelengths
\\
\hline

128
&
aperture\_array/element\_pattern/enable\_numerical
&
If \textbf{true}, make use of any available numerical element pattern files. If numerical pattern data are missing, the functional type will be used instead.
&
Bool
&
true
\\
\hline

129
&
aperture\_array/element\_pattern/taper/type
&
The type of tapering function to apply to the element pattern.
&
Allowed values
&
None
\\
\hline

130
&
aperture\_array/element\_pattern/taper/cosine\_power
&
If a cosine element taper is selected, this setting gives the power of the cosine(theta) function.
&
Double
&
1.0
\\
\hline

131
&
aperture\_array/element\_pattern/taper/gaussian\_fwhm\_deg
&
If a Gaussian element taper is selected, this setting gives the full-width half maximum value of the Gaussian, in degrees.
&
Double
&
45.0
\\
\hline

132
&
gaussian\_beam/fwhm\_deg
&
For stations using a simple Gaussian beam, this setting gives the full-width half maximum value of the Gaussian station beam at the reference frequency, in degrees.
&
Double
&
1.0
\\
\hline

133
&
gaussian\_beam/ref\_freq\_hz
&
The reference frequency of the specified FWHM, in Hz.
&
Double
&
0.0
\\
\hline

134
&
output\_directory
&
Path used to save the final telescope model directory, excluding any element pattern data (useful for debugging). Leave blank if not required.
&
Path name
&

\\
\hline

\end{longtable}
\end{center}
\normalsize
\newpage
