%
% This is an autogenerated file - do not edit!
%


\fontsize{8}{10}\selectfont
\begin{center}
\begin{longtable}{|c|L{9cm}|L{8.5cm}|L{3.0cm}|L{1.7cm}|}

\hline
  \rowcolor{lightgray}
  {\textbf{ID}} &
  {\textbf{Key}} &
  {\textbf{Description}} &
  {\textbf{Allowed values}} &
  {\textbf{Default}} \\[0.5ex] \hline
\endfirsthead
\hline
  \rowcolor{lightgray}
  {\textbf{ID}} &
  {\textbf{Key}} &
  {\textbf{Description}} &
  {\textbf{Allowed values}} &
  {\textbf{Default}} \\[0.5ex] \hline
\endhead
  \multicolumn{5}{l}{{Continued on next page\ldots}} \\
\endfoot
 
\endlastfoot
184
&
fov\_deg
&
Total field of view in degrees.
&
Unsigned double
&
2.0
\\
\hline

185
&
size
&
Image width in one dimension (e.g. a value of 256 would give a 256 by 256 image).
&
Integer > 0
&
256
\\
\hline

186
&
image\_type
&
The type of image to generate. Note that the Stokes parameter images (if selected) are uncalibrated, and are formed simply using the standard combinations of the linear polarisations: <ul> <li>I = 0.5 (XX + YY)</li> <li>Q = 0.5 (XX - YY)</li> <li>U = 0.5 (XY + YX)</li> <li>V = -0.5i (XY - YX)</li> </ul> The point spread function of the observation can be generated using the PSF option.
&
Allowed values
&
I
\\
\hline

187
&
channel\_snapshots
&
If true, then produce an image cube containing snapshots for each frequency channel. If false, then use frequency-synthesis to stack the channels in the final image.
&
Bool
&
true
\\
\hline

188
&
channel\_start
&
The start channel index to include in the image or image cube.
&
Unsigned int
&
0
\\
\hline

189
&
channel\_end
&
The end channel index to include in the image or image cube.
&
Allowed values
&
max
\\
\hline

190
&
time\_snapshots
&
If true, then produce an image cube containing snapshots for each time step. If false, then use time-synthesis to stack the times in the final image.
&
Bool
&
true
\\
\hline

191
&
time\_start
&
The start time index to include in the image or image cube.
&
Unsigned int
&
0
\\
\hline

192
&
time\_end
&
The end time index to include in the image or image cube.
&
Allowed values
&
max
\\
\hline

193
&
direction
&
Specifies the direction of the image phase centre. <ul> <li>If `Observation direction' is selected, the image is centred on the pointing direction of the primary beam.</li> <li>If `RA, Dec.' is selected, the image is centred on the values of RA and Dec. found below.</li> </ul>
&
Allowed values
&
Observation
\\
\hline

194
&
direction/ra\_deg
&
The Right Ascension of the image phase centre. This value is used if the image centre direction is set to `RA, Dec.'.
&
Double
&
0.0
\\
\hline

195
&
direction/dec\_deg
&
The Declination of the image phase centre. This value is used if the image centre direction is set to `RA, Dec.'.
&
Double
&
0.0
\\
\hline

196
&
input\_vis\_data
&
Path to the input OSKAR visibility data file.
&
Path name
&
None
\\
\hline

197
&
root\_path
&
Path consisting of the root of the image filename used to save the output image. The full filename will be constructed as {\texttt{ \textbf{ $<$root$>$\_$<$image\_type$>$.$<$extension$>$, }}}
&
Path name
&
None
\\
\hline

198
&
fits\_image
&
If true, save the image in FITS format.
&
Bool
&
true
\\
\hline

199
&
oskar\_image
&
If true, save the image in OSKAR image binary format.
&
Bool
&
true
\\
\hline

200
&
overwrite
&
If \textbf{true}, existing image files will be overwritten. If \textbf{false}, new image files of the same name will be created by appending an number to the existing filename with the pattern: <br />{\texttt{ \textbf{ $<$filename$>$-$<$N$>$.$<$extension$>$, }}}<br /> where N starts at 1 and is incremented for each new image created.
&
Bool
&
true
\\
\hline

\end{longtable}
\end{center}
\normalsize
\newpage
