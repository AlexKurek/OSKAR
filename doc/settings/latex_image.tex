%
% This is an autogenerated file - do not edit!
%


\fontsize{8}{10}\selectfont
\begin{center}
\begin{longtable}{|L{9cm}|L{8.5cm}|L{4.0cm}|L{1.7cm}|}

\hline
  \rowcolor{lightgray}
  {\textbf{Key}} &
  {\textbf{Description}} &
  {\textbf{Allowed values}} &
  {\textbf{Default}} \\ \hline
\endfirsthead
\hline
  \rowcolor{lightgray}
  {\textbf{Key}} &
  {\textbf{Description}} &
  {\textbf{Allowed values}} &
  {\textbf{Default}} \\ \hline
\endhead
  \multicolumn{4}{l}{{Continued on next page\ldots}} \\
\endfoot
 
\endlastfoot
fov\_deg
&
{Total field of view in degrees.}
&
Unsigned double
&
2.0
\\
\hline

size
&
{Image width in one dimension (e.g. a value of 256 would give a 256 by 256 image).}
&
Integer > 0
&
256
\\
\hline

image\_type
&
{The type of image to generate. Note that the Stokes parameter images (if selected) are uncalibrated, and are formed simply using the standard combinations of the linear polarisations: 
{\begin{itemize}[leftmargin=5ex, topsep=0pt, partopsep=0pt, itemsep=4pt, parsep=0pt]
\vspace{8pt} \item {I = 0.5 (XX + YY)}
 \item {Q = 0.5 (XX - YY)}
 \item {U = 0.5 (XY + YX)}
 \item {V = -0.5i (XY - YX)}
 \vspace{8pt}
\end{itemize}}
 The point spread function of the observation can be generated using the PSF option.}
&
{One of the following:}
{\begin{itemize}[leftmargin=5ex, topsep=0pt, partopsep=0pt, itemsep=2pt, parsep=0pt]
\vspace{4pt}\item {Linear (XX,XY,YX,YY)}
\item {XX}
\item {XY}
\item {YX}
\item {YY}
\item {Stokes (I,Q,U,V)}
\item {I}
\item {Q}
\item {U}
\item {V}
\item {PSF}
\end{itemize}
}
&
I
\\
\hline

channel\_snapshots
&
{If true, then produce an image cube containing snapshots for each frequency channel. If false, then use frequency-synthesis to stack the channels in the final image.}
&
Bool
&
true
\\
\hline

channel\_start
&
{The start channel index to include in the image or image cube.}
&
Unsigned integer
&
0
\\
\hline

channel\_end
&
{The end channel index to include in the image or image cube.}
&
{Integer $\geq$ 0, or `max'}
&
max
\\
\hline

time\_snapshots
&
{If true, then produce an image cube containing snapshots for each time step. If false, then use time-synthesis to stack the times in the final image.}
&
Bool
&
true
\\
\hline

time\_start
&
{The start time index to include in the image or image cube.}
&
Unsigned integer
&
0
\\
\hline

time\_end
&
{The end time index to include in the image or image cube.}
&
{Integer $\geq$ 0, or `max'}
&
max
\\
\hline

direction
&
{Specifies the direction of the image phase centre. 
{\begin{itemize}[leftmargin=5ex, topsep=0pt, partopsep=0pt, itemsep=4pt, parsep=0pt]
\vspace{8pt} \item {If \textbf{Observation direction} is selected, the image is centred on the pointing direction of the primary beam.}
 \item {If \textbf{RA, Dec.} is selected, the image is centred on the values of RA and Dec. found below.}
 \vspace{8pt}
\end{itemize}}
}
&
{One of the following:}
{\begin{itemize}[leftmargin=5ex, topsep=0pt, partopsep=0pt, itemsep=2pt, parsep=0pt]
\vspace{4pt}\item {Observation direction}
\item {RA, Dec.}
\end{itemize}
}
&
Observation
\\
\hline

direction/ra\_deg
&
{The Right Ascension of the image phase centre. This value is used if the image centre direction is set to `RA, Dec.'.}
&
Double
&
0.0
\\
\hline

direction/dec\_deg
&
{The Declination of the image phase centre. This value is used if the image centre direction is set to `RA, Dec.'.}
&
Double
&
0.0
\\
\hline

input\_vis\_data
&
{Path to the input OSKAR visibility data file.}
&
Path name
&
None
\\
\hline

root\_path
&
{Path consisting of the root of the image filename used to save the output image. The full filename will be constructed as {\texttt{ \textbf{ $<$root$>$\_$<$image\_type$>$.$<$extension$>$ }}}}
&
Path name
&
None
\\
\hline

fits\_image
&
{If true, save the image in FITS format.}
&
Bool
&
true
\\
\hline

oskar\_image
&
{If true, save the image in OSKAR image binary format.}
&
Bool
&
true
\\
\hline

overwrite
&
{If \textbf{true}, existing image files will be overwritten. If \textbf{false}, new image files of the same name will be created by appending an number to the existing filename with the pattern: 
\vspace{8pt}\par\noindent{\texttt{ \textbf{ $<$filename$>$-$<$N$>$.$<$extension$>$, }}}
\vspace{8pt}\par\noindent where N starts at 1 and is incremented for each new image created.}
&
Bool
&
true
\\
\hline

\end{longtable}
\end{center}
\normalsize
\newpage
