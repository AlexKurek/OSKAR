%
% This is an autogenerated file - do not edit!
%


\fontsize{8}{10}\selectfont
\begin{center}
\begin{longtable}{|c|L{9cm}|L{8.5cm}|L{3.0cm}|L{1.7cm}|}

\hline
  \rowcolor{lightgray}
  {\textbf{ID}} &
  {\textbf{Key}} &
  {\textbf{Description}} &
  {\textbf{Allowed values}} &
  {\textbf{Default}} \\[0.5ex] \hline
\endfirsthead
\hline
  \rowcolor{lightgray}
  {\textbf{ID}} &
  {\textbf{Key}} &
  {\textbf{Description}} &
  {\textbf{Allowed values}} &
  {\textbf{Default}} \\[0.5ex] \hline
\endhead
  \multicolumn{5}{l}{{Continued on next page\ldots}} \\
\endfoot
 
\endlastfoot
135
&
channel\_bandwidth\_hz
&
The channel width, in Hz, used to simulate bandwidth smearing. (Note that this can be different to the frequency increment if channels do not cover a contiguous frequency range.)
&
Double
&
0.0
\\
\hline

136
&
time\_average\_sec
&
The correlator time-average duration, in seconds, used to simulate time averaging smearing.
&
Double
&
0.0
\\
\hline

137
&
num\_vis\_ave
&
Number of averaged evaluations of the full Measurement Equation per visibility dump.
&
Integer > 0
&
1
\\
\hline

138
&
num\_fringe\_ave
&
Number of averaged evaluations of the K-Jones matrix per Measurement Equation average.
&
Integer > 0
&
1
\\
\hline

139
&
uv\_filter\_min
&
The minimum value of the baseline UV length allowed by the filter. \textbf{Note that visibilities on baseline UV lengths outside this range will not be evaluated!}
&
Double in range 0 $\leq$ value $\leq$ MAX or `min'
&
min
\\
\hline

140
&
uv\_filter\_max
&
The maximum value of the baseline UV length allowed by the filter. \textbf{Note that visibilities on baseline UV lengths outside this range will not be evaluated!}
&
Double in range 0 $\leq$ value $\leq$ MAX or `max'
&
max
\\
\hline

141
&
uv\_filter\_units
&
The units of the baseline UV length filter values.
&
Allowed values
&
Wavelengths
\\
\hline

142
&
use\_common\_sky
&
If \textbf{true}, then use a short baseline approximation where source positions are the same relative to every station. If \textbf{false}, then re-evaluate all source positions and all station beams.
&
Bool
&
true
\\
\hline

143
&
noise/enable
&
If \textbf{true}, noise addition is enabled.
&
Bool
&
false
\\
\hline

144
&
noise/seed
&
Random number generator seed.
&
`time' or integer seed $\geq$ 1
&
1
\\
\hline

145
&
noise/freq
&
Specification of the list of frequencies at which noise values are defined: <ul> <li>\textbf{Telescope model}: frequencies are loaded from the data file in the telescope model directory.</li> <li>\textbf{Observation settings}: frequencies are defined by the observation settings.</li> <li>\textbf{Data file}: frequencies are loaded from the specified data file.</li> <li>\textbf{Range}: frequencies are specified by the range parameters.</li> </ul>
&
Allowed values
&
Telescope
\\
\hline

146
&
noise/freq/file
&
Data file consisting of an ASCII list of frequencies.
&
Path name
&

\\
\hline

147
&
noise/freq/number
&
Number of frequencies.
&
Unsigned int
&
0
\\
\hline

148
&
noise/freq/start
&
Start frequency, in Hz.
&
Unsigned double
&
0
\\
\hline

149
&
noise/freq/inc
&
Frequency increment, in Hz.
&
Unsigned double
&
0
\\
\hline

150
&
noise/values
&
Single polarisation noise value specification type: <ul> <li>\textbf{Telescope model priority}: values are loaded from files in the telescope model directory, according to the default file type priority.</li> <li>\textbf{RMS flux density}: use values specified in terms of noise RMS flux density. </li> <li>\textbf{Sensitivity}: use values specified in terms of station sensitivity.</li> <li>\textbf{Temperature ...}: use values specified by the system temperature, effective area, and system efficiency.</li> </ul> <i>Note: Noise values are interpreted as a function of frequency. The list of frequencies to which noise values correspond is based upon the value of the noise frequency specification.</i>.
&
Allowed values
&
Telescope
\\
\hline

151
&
noise/values/rms
&
Root mean square (RMS) flux density specification: <ul> <li>\textbf{No override}: values are loaded from RMS files found in the telescope model directory.</li> <li>\textbf{Data file}: values are loaded from the specified file.</li> <li>\textbf{Range}: values are evaluated according to the specified range parameters.</li> </ul>
&
Allowed values
&
No
\\
\hline

152
&
noise/values/rms/file
&
RMS flux density data file.
&
Path name
&
None
\\
\hline

153
&
noise/values/rms/start
&
RMS flux density range start value, in Jy.
&
Double
&
0.0
\\
\hline

154
&
noise/values/rms/end
&
RMS flux density range end value, in Jy.
&
Double
&
0.0
\\
\hline

155
&
noise/values/sensitivity
&
System sensitivity (or System Equivalent Flux density, SEFD) specification type: <ul> <li>\textbf{No override}: values are loaded from sensitivity files found in the telescope model directory.</li> <li>\textbf{Data file}: values are loaded from the specified file.</li> <li>\textbf{Range}: values are evaluated according to the specified range parameters.</li> </ul>
&
Allowed values
&
No
\\
\hline

156
&
noise/values/sensitivity/file
&
Data file containing system sensitivity value(s).
&
Path name
&
None
\\
\hline

157
&
noise/values/sensitivity/start
&
Sensitivity range start value, in Jy.
&
Double
&
0.0
\\
\hline

158
&
noise/values/sensitivity/end
&
Sensitivity range end value, in Jy.
&
Double
&
0.0
\\
\hline

159
&
noise/values/components/t\_sys
&
System temperature specification type: <ul> <li>\textbf{No override}: values are loaded from system temperature files found in the telescope model directory.</li> <li>\textbf{Data file}: values are loaded from the specified file.</li> <li>\textbf{Range}: values are evaluated according to the specified range parameters.</li> </ul>
&
Allowed values
&
No
\\
\hline

160
&
noise/values/components/t\_sys/file
&
Data file containing system temperature value(s).
&
Path name
&
None
\\
\hline

161
&
noise/values/components/t\_sys/start
&
System temperature range start value, in K.
&
Double
&
0.0
\\
\hline

162
&
noise/values/components/t\_sys/end
&
System temperature range end value, in K.
&
Double
&
0.0
\\
\hline

163
&
noise/values/components/area
&
Station effective area specification type <ul> <li>\textbf{No override}: values are loaded from system temperature files found in the telescope model directory.</li> <li>\textbf{Data file}: values are loaded from the specified file.</li> <li>\textbf{Range}: values are evaluated according to the specified range parameters.</li> </ul>
&
Allowed values
&
No
\\
\hline

164
&
noise/values/components/area/file
&
Data file containing effective area value(s).
&
Path name
&
None
\\
\hline

165
&
noise/values/components/area/start
&
Effective area range start value, in m<sup>2</sup>.
&
Double
&
0.0
\\
\hline

166
&
noise/values/components/area/end
&
Effective area range end value, in m<sup>2</sup>.
&
Double
&
0.0
\\
\hline

167
&
noise/values/components/efficiency
&
Station system efficiency specification type: <ul> <li>\textbf{No override}: values are loaded from system temperature files found in the telescope model directory.</li> <li>\textbf{Data file}: values are loaded from the specified file.</li> <li>\textbf{Range}: values are evaluated according to the specified range parameters.</li> </ul>
&
Allowed values
&
No
\\
\hline

168
&
noise/values/components/efficiency/file
&
Data file containing system efficiency value(s).
&
Path name
&
None
\\
\hline

169
&
noise/values/components/efficiency/start
&
System efficiency range start value (allowed range: 0.0 to 1.0).
&
Double. 0 $\leq$ value $\leq$ 1
&
0.0
\\
\hline

170
&
noise/values/components/efficiency/end
&
System efficiency range end value (allowed range: 0.0 to 1.0).
&
Double. 0 $\leq$ value $\leq$ 1
&
0.0
\\
\hline

171
&
oskar\_vis\_filename
&
Path of the OSKAR visibility output file containing the results of the simulation. Leave blank if not required.
&
Path name
&

\\
\hline

172
&
ms\_filename
&
Path of the Measurement Set containing the results of the simulation. Leave blank if not required.
&
Path name
&

\\
\hline

\end{longtable}
\end{center}
\normalsize
\newpage
