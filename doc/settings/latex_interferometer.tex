%
% This is an autogenerated file - do not edit!
%


\fontsize{8}{10}\selectfont
\begin{center}
\begin{longtable}{|c|L{9cm}|L{7.5cm}|L{4.0cm}|L{1.7cm}|}

\hline
  \rowcolor{lightgray}
  {\textbf{ID}} &
  {\textbf{Key}} &
  {\textbf{Description}} &
  {\textbf{Allowed values}} &
  {\textbf{Default}} \\ \hline
\endfirsthead
\hline
  \rowcolor{lightgray}
  {\textbf{ID}} &
  {\textbf{Key}} &
  {\textbf{Description}} &
  {\textbf{Allowed values}} &
  {\textbf{Default}} \\ \hline
\endhead
  \multicolumn{5}{l}{{Continued on next page\ldots}} \\
\endfoot
 
\endlastfoot
145
&
channel\_bandwidth\_hz
&
{The channel width, in Hz, used to simulate bandwidth smearing. (Note that this can be different to the frequency increment if channels do not cover a contiguous frequency range.)}
&
Double
&
0.0
\\
\hline

146
&
time\_average\_sec
&
{The correlator time-average duration, in seconds, used to simulate time averaging smearing.}
&
Double
&
0.0
\\
\hline

147
&
num\_vis\_ave
&
{Number of averaged evaluations of the full Measurement Equation per visibility dump.}
&
Integer > 0
&
1
\\
\hline

148
&
num\_fringe\_ave
&
{Number of averaged evaluations of the K-Jones matrix per Measurement Equation average.}
&
Integer > 0
&
1
\\
\hline

149
&
uv\_filter\_min
&
{The minimum value of the baseline UV length allowed by the filter. \textbf{Note that visibilities on baseline UV lengths outside this range will not be evaluated!}}
&
`min' or double in range 0 $\leq$ {\textbf{$x$}} $\leq$ MAX
&
min
\\
\hline

150
&
uv\_filter\_max
&
{The maximum value of the baseline UV length allowed by the filter. \textbf{Note that visibilities on baseline UV lengths outside this range will not be evaluated!}}
&
`max' or double in range 0 $\leq$ {\textbf{$x$}} $\leq$ MAX
&
max
\\
\hline

151
&
uv\_filter\_units
&
{The units of the baseline UV length filter values.}
&
{One of the following:}
{\begin{itemize}[leftmargin=2ex, topsep=0pt]
\itemsep1pt \parskip0pt \parsep0pt 
\item {Wavelengths}
\item {Metres}
\end{itemize}
\vspace{-\baselineskip}\mbox{}
}
&
Wavelengths
\\
\hline

152
&
use\_common\_sky
&
{If \textbf{true}, then use a short baseline approximation where source positions are the same relative to every station. If \textbf{false}, then re-evaluate all source positions and all station beams.}
&
Bool
&
true
\\
\hline

153
&
noise/enable
&
{If \textbf{true}, noise addition is enabled.}
&
Bool
&
false
\\
\hline

154
&
noise/seed
&
{Random number generator seed.}
&
`time' or integer seed $\geq$ 1
&
1
\\
\hline

155
&
noise/freq
&
{Specification of the list of frequencies at which noise values are defined: {\begin{itemize}
 \item {\textbf{Telescope model}: frequencies are loaded from the data file in the telescope model directory.}
 \item {\textbf{Observation settings}: frequencies are defined by the observation settings.}
 \item {\textbf{Data file}: frequencies are loaded from the specified data file.}
 \item {\textbf{Range}: frequencies are specified by the range parameters.}
 \end{itemize}}
}
&
{One of the following:}
{\begin{itemize}[leftmargin=2ex, topsep=0pt]
\itemsep1pt \parskip0pt \parsep0pt 
\item {Telescope model}
\item {Observation settings}
\item {Data file}
\item {Range}
\end{itemize}
\vspace{-\baselineskip}\mbox{}
}
&
Telescope
\\
\hline

156
&
noise/freq/file
&
{Data file consisting of an ASCII list of frequencies.}
&
Path name
&

\\
\hline

157
&
noise/freq/number
&
{Number of frequencies.}
&
Unsigned integer
&
0
\\
\hline

158
&
noise/freq/start
&
{Start frequency, in Hz.}
&
Unsigned double
&
0
\\
\hline

159
&
noise/freq/inc
&
{Frequency increment, in Hz.}
&
Unsigned double
&
0
\\
\hline

160
&
noise/values
&
{Single polarisation noise value specification type: {\begin{itemize}
 \item {\textbf{Telescope model priority}: values are loaded from files in the telescope model directory, according to the default file type priority.}
 \item {\textbf{RMS flux density}: use values specified in terms of noise RMS flux density. }
 \item {\textbf{Sensitivity}: use values specified in terms of station sensitivity.}
 \item {\textbf{Temperature ...}: use values specified by the system temperature, effective area, and system efficiency.}
 \end{itemize}}
 \textit{Note: Noise values are interpreted as a function of frequency. The list of frequencies to which noise values correspond is based upon the value of the noise frequency specification.}.}
&
{One of the following:}
{\begin{itemize}[leftmargin=2ex, topsep=0pt]
\itemsep1pt \parskip0pt \parsep0pt 
\item {Telescope model priority}
\item {RMS flux density Sensitivity}
\item {Temperature, area, and system efficiency}
\end{itemize}
\vspace{-\baselineskip}\mbox{}
}
&
Telescope
\\
\hline

161
&
noise/values/rms
&
{Root mean square (RMS) flux density specification: {\begin{itemize}
 \item {\textbf{No override}: values are loaded from RMS files found in the telescope model directory.}
 \item {\textbf{Data file}: values are loaded from the specified file.}
 \item {\textbf{Range}: values are evaluated according to the specified range parameters.}
 \end{itemize}}
}
&
{One of the following:}
{\begin{itemize}[leftmargin=2ex, topsep=0pt]
\itemsep1pt \parskip0pt \parsep0pt 
\item {No override (telescope model)}
\item {Data file}
\item {Range}
\end{itemize}
\vspace{-\baselineskip}\mbox{}
}
&
No
\\
\hline

162
&
noise/values/rms/file
&
{RMS flux density data file.}
&
Path name
&
None
\\
\hline

163
&
noise/values/rms/start
&
{RMS flux density range start value, in Jy.}
&
Double
&
0.0
\\
\hline

164
&
noise/values/rms/end
&
{RMS flux density range end value, in Jy.}
&
Double
&
0.0
\\
\hline

165
&
noise/values/sensitivity
&
{System sensitivity (or System Equivalent Flux density, SEFD) specification type: {\begin{itemize}
 \item {\textbf{No override}: values are loaded from sensitivity files found in the telescope model directory.}
 \item {\textbf{Data file}: values are loaded from the specified file.}
 \item {\textbf{Range}: values are evaluated according to the specified range parameters.}
 \end{itemize}}
}
&
{One of the following:}
{\begin{itemize}[leftmargin=2ex, topsep=0pt]
\itemsep1pt \parskip0pt \parsep0pt 
\item {No override (telescope model)}
\item {Data file}
\item {Range}
\end{itemize}
\vspace{-\baselineskip}\mbox{}
}
&
No
\\
\hline

166
&
noise/values/sensitivity/file
&
{Data file containing system sensitivity value(s).}
&
Path name
&
None
\\
\hline

167
&
noise/values/sensitivity/start
&
{Sensitivity range start value, in Jy.}
&
Double
&
0.0
\\
\hline

168
&
noise/values/sensitivity/end
&
{Sensitivity range end value, in Jy.}
&
Double
&
0.0
\\
\hline

169
&
noise/values/components/t\_sys
&
{System temperature specification type: {\begin{itemize}
 \item {\textbf{No override}: values are loaded from system temperature files found in the telescope model directory.}
 \item {\textbf{Data file}: values are loaded from the specified file.}
 \item {\textbf{Range}: values are evaluated according to the specified range parameters.}
 \end{itemize}}
}
&
{One of the following:}
{\begin{itemize}[leftmargin=2ex, topsep=0pt]
\itemsep1pt \parskip0pt \parsep0pt 
\item {No override (telescope model)}
\item {Data file}
\item {Range}
\end{itemize}
\vspace{-\baselineskip}\mbox{}
}
&
No
\\
\hline

170
&
noise/values/components/t\_sys/file
&
{Data file containing system temperature value(s).}
&
Path name
&
None
\\
\hline

171
&
noise/values/components/t\_sys/start
&
{System temperature range start value, in K.}
&
Double
&
0.0
\\
\hline

172
&
noise/values/components/t\_sys/end
&
{System temperature range end value, in K.}
&
Double
&
0.0
\\
\hline

173
&
noise/values/components/area
&
{Station effective area specification type {\begin{itemize}
 \item {\textbf{No override}: values are loaded from system temperature files found in the telescope model directory.}
 \item {\textbf{Data file}: values are loaded from the specified file.}
 \item {\textbf{Range}: values are evaluated according to the specified range parameters.}
 \end{itemize}}
}
&
{One of the following:}
{\begin{itemize}[leftmargin=2ex, topsep=0pt]
\itemsep1pt \parskip0pt \parsep0pt 
\item {No override (telescope model)}
\item {Data file}
\item {Range}
\end{itemize}
\vspace{-\baselineskip}\mbox{}
}
&
No
\\
\hline

174
&
noise/values/components/area/file
&
{Data file containing effective area value(s).}
&
Path name
&
None
\\
\hline

175
&
noise/values/components/area/start
&
{Effective area range start value, in m<sup>2</sup>.}
&
Double
&
0.0
\\
\hline

176
&
noise/values/components/area/end
&
{Effective area range end value, in m<sup>2</sup>.}
&
Double
&
0.0
\\
\hline

177
&
noise/values/components/efficiency
&
{Station system efficiency specification type: {\begin{itemize}
 \item {\textbf{No override}: values are loaded from system temperature files found in the telescope model directory.}
 \item {\textbf{Data file}: values are loaded from the specified file.}
 \item {\textbf{Range}: values are evaluated according to the specified range parameters.}
 \end{itemize}}
}
&
{One of the following:}
{\begin{itemize}[leftmargin=2ex, topsep=0pt]
\itemsep1pt \parskip0pt \parsep0pt 
\item {No override (telescope model)}
\item {Data file}
\item {Range}
\end{itemize}
\vspace{-\baselineskip}\mbox{}
}
&
No
\\
\hline

178
&
noise/values/components/efficiency/file
&
{Data file containing system efficiency value(s).}
&
Path name
&
None
\\
\hline

179
&
noise/values/components/efficiency/start
&
{System efficiency range start value (allowed range: 0.0 to 1.0).}
&
Double in range 0 $\leq$ $x$ $\leq$ 1
&
0.0
\\
\hline

180
&
noise/values/components/efficiency/end
&
{System efficiency range end value (allowed range: 0.0 to 1.0).}
&
Double in range 0 $\leq$ $x$ $\leq$ 1
&
0.0
\\
\hline

181
&
oskar\_vis\_filename
&
{Path of the OSKAR visibility output file containing the results of the simulation. Leave blank if not required.}
&
Path name
&

\\
\hline

182
&
ms\_filename
&
{Path of the Measurement Set containing the results of the simulation. Leave blank if not required.}
&
Path name
&

\\
\hline

\end{longtable}
\end{center}
\normalsize
\newpage
