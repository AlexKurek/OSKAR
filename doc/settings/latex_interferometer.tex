%
% This is an autogenerated file - do not edit!
%


\fontsize{8}{10}\selectfont
\begin{center}
\begin{longtable}{|L{9cm}|L{8.5cm}|L{4.0cm}|L{1.7cm}|}

\hline
  \rowcolor{lightgray}
  {\textbf{Key}} &
  {\textbf{Description}} &
  {\textbf{Allowed values}} &
  {\textbf{Default}} \\ \hline
\endfirsthead
\hline
  \rowcolor{lightgray}
  {\textbf{Key}} &
  {\textbf{Description}} &
  {\textbf{Allowed values}} &
  {\textbf{Default}} \\ \hline
\endhead
  \multicolumn{4}{l}{{Continued on next page\ldots}} \\
\endfoot
 
\endlastfoot
channel\_bandwidth\_hz
&
{The channel width, in Hz, used to simulate bandwidth smearing. (Note that this can be different to the frequency increment if channels do not cover a contiguous frequency range.)}
&
Double
&
0.0
\\
\hline

time\_average\_sec
&
{The correlator time-average duration, in seconds, used to simulate time averaging smearing.}
&
Double
&
0.0
\\
\hline

max\_time\_samples\_per\_block
&
{The maximum number of time samples held in memory before being flushed to disk.}
&
Int
&
10
\\
\hline

correlation\_type
&
{The type of correlations to produce: either cross-correlations, auto-correlations, or both.}
&
{One of the following:}
{\begin{itemize}[leftmargin=5ex, topsep=0pt, partopsep=0pt, itemsep=2pt, parsep=0pt]
\vspace{4pt}\item {Cross-correlations}
\item {Auto-correlations}
\item {Both}
\end{itemize}
}
&
Cross-correlations
\\
\hline

uv\_filter\_min
&
{The minimum value of the baseline UV length allowed by the filter. \textbf{Note that visibilities on baseline UV lengths outside this range will not be evaluated!}}
&
{Double $\geq$ 0, or `min'}
&
min
\\
\hline

uv\_filter\_max
&
{The maximum value of the baseline UV length allowed by the filter. \textbf{Note that visibilities on baseline UV lengths outside this range will not be evaluated!}}
&
{Double $\geq$ 0, or `max'}
&
max
\\
\hline

uv\_filter\_units
&
{The units of the baseline UV length filter values.}
&
{One of the following:}
{\begin{itemize}[leftmargin=5ex, topsep=0pt, partopsep=0pt, itemsep=2pt, parsep=0pt]
\vspace{4pt}\item {Wavelengths}
\item {Metres}
\end{itemize}
}
&
Wavelengths
\\
\hline

noise/enable
&
{If \textbf{true}, noise addition is enabled.}
&
Bool
&
false
\\
\hline

noise/seed
&
{Random number generator seed.}
&
Integer $\geq$ 1, or `time'
&
1
\\
\hline

noise/freq
&
{Specification of the list of frequencies at which noise values are defined: 
{\begin{itemize}[leftmargin=5ex, topsep=0pt, partopsep=0pt, itemsep=4pt, parsep=0pt]
\vspace{8pt}
 \item {\textbf{Telescope model}: frequencies are loaded from the data file in the telescope model directory.}
 \item {\textbf{Observation settings}: frequencies are defined by the observation settings.}
 \item {\textbf{Data file}: frequencies are loaded from the specified data file.}
 \item {\textbf{Range}: frequencies are specified by the range parameters.}
 \vspace{8pt}
\end{itemize}}
}
&
{One of the following:}
{\begin{itemize}[leftmargin=5ex, topsep=0pt, partopsep=0pt, itemsep=2pt, parsep=0pt]
\vspace{4pt}\item {Telescope model}
\item {Observation settings}
\item {Data file}
\item {Range}
\end{itemize}
}
&
Telescope
\\
\hline

noise/freq/file
&
{Data file consisting of an ASCII list of frequencies.}
&
Path name
&

\\
\hline

noise/freq/number
&
{Number of frequencies.}
&
Unsigned integer
&
0
\\
\hline

noise/freq/start
&
{Start frequency, in Hz.}
&
Unsigned double
&
0
\\
\hline

noise/freq/inc
&
{Frequency increment, in Hz.}
&
Unsigned double
&
0
\\
\hline

noise/values
&
{Single polarisation noise value specification type: 
{\begin{itemize}[leftmargin=5ex, topsep=0pt, partopsep=0pt, itemsep=4pt, parsep=0pt]
\vspace{8pt}
 \item {\textbf{Telescope model priority}: values are loaded from files in the telescope model directory, according to the default file type priority.}
 \item {\textbf{RMS flux density}: use values specified in terms of noise RMS flux density. }
 \item {\textbf{Sensitivity}: use values specified in terms of station sensitivity.}
 \item {\textbf{Temperature ...}: use values specified by the system temperature, effective area, and system efficiency.}
 \vspace{8pt}
\end{itemize}}
 \textit{Note: Noise values are interpreted as a function of frequency. The list of frequencies to which noise values correspond is based upon the value of the noise frequency specification.}.}
&
{One of the following:}
{\begin{itemize}[leftmargin=5ex, topsep=0pt, partopsep=0pt, itemsep=2pt, parsep=0pt]
\vspace{4pt}\item {Telescope model priority}
\item {RMS flux density}
\item {Sensitivity}
\item {Temperature, area, and system efficiency}
\end{itemize}
}
&
Telescope
\\
\hline

noise/values/rms
&
{Root mean square (RMS) flux density specification: 
{\begin{itemize}[leftmargin=5ex, topsep=0pt, partopsep=0pt, itemsep=4pt, parsep=0pt]
\vspace{8pt}
 \item {\textbf{No override}: values are loaded from RMS files found in the telescope model directory.}
 \item {\textbf{Data file}: values are loaded from the specified file.}
 \item {\textbf{Range}: values are evaluated according to the specified range parameters.}
 \vspace{8pt}
\end{itemize}}
}
&
{One of the following:}
{\begin{itemize}[leftmargin=5ex, topsep=0pt, partopsep=0pt, itemsep=2pt, parsep=0pt]
\vspace{4pt}\item {No override (telescope model)}
\item {Data file}
\item {Range}
\end{itemize}
}
&
No
\\
\hline

noise/values/rms/file
&
{RMS flux density data file.}
&
Path name
&
None
\\
\hline

noise/values/rms/start
&
{RMS flux density range start value, in Jy.}
&
Double
&
0.0
\\
\hline

noise/values/rms/end
&
{RMS flux density range end value, in Jy.}
&
Double
&
0.0
\\
\hline

noise/values/sensitivity
&
{System sensitivity (or System Equivalent Flux density, SEFD) specification type: 
{\begin{itemize}[leftmargin=5ex, topsep=0pt, partopsep=0pt, itemsep=4pt, parsep=0pt]
\vspace{8pt}
 \item {\textbf{No override}: values are loaded from sensitivity files found in the telescope model directory.}
 \item {\textbf{Data file}: values are loaded from the specified file.}
 \item {\textbf{Range}: values are evaluated according to the specified range parameters.}
 \vspace{8pt}
\end{itemize}}
}
&
{One of the following:}
{\begin{itemize}[leftmargin=5ex, topsep=0pt, partopsep=0pt, itemsep=2pt, parsep=0pt]
\vspace{4pt}\item {No override (telescope model)}
\item {Data file}
\item {Range}
\end{itemize}
}
&
No
\\
\hline

noise/values/sensitivity/file
&
{Data file containing system sensitivity value(s).}
&
Path name
&
None
\\
\hline

noise/values/sensitivity/start
&
{Sensitivity range start value, in Jy.}
&
Double
&
0.0
\\
\hline

noise/values/sensitivity/end
&
{Sensitivity range end value, in Jy.}
&
Double
&
0.0
\\
\hline

noise/values/components/t\_sys
&
{System temperature specification type: 
{\begin{itemize}[leftmargin=5ex, topsep=0pt, partopsep=0pt, itemsep=4pt, parsep=0pt]
\vspace{8pt}
 \item {\textbf{No override}: values are loaded from system temperature files found in the telescope model directory.}
 \item {\textbf{Data file}: values are loaded from the specified file.}
 \item {\textbf{Range}: values are evaluated according to the specified range parameters.}
 \vspace{8pt}
\end{itemize}}
}
&
{One of the following:}
{\begin{itemize}[leftmargin=5ex, topsep=0pt, partopsep=0pt, itemsep=2pt, parsep=0pt]
\vspace{4pt}\item {No override (telescope model)}
\item {Data file}
\item {Range}
\end{itemize}
}
&
No
\\
\hline

noise/values/components/t\_sys/file
&
{Data file containing system temperature value(s).}
&
Path name
&
None
\\
\hline

noise/values/components/t\_sys/start
&
{System temperature range start value, in K.}
&
Double
&
0.0
\\
\hline

noise/values/components/t\_sys/end
&
{System temperature range end value, in K.}
&
Double
&
0.0
\\
\hline

noise/values/components/area
&
{Station effective area specification type 
{\begin{itemize}[leftmargin=5ex, topsep=0pt, partopsep=0pt, itemsep=4pt, parsep=0pt]
\vspace{8pt}
 \item {\textbf{No override}: values are loaded from system temperature files found in the telescope model directory.}
 \item {\textbf{Data file}: values are loaded from the specified file.}
 \item {\textbf{Range}: values are evaluated according to the specified range parameters.}
 \vspace{8pt}
\end{itemize}}
}
&
{One of the following:}
{\begin{itemize}[leftmargin=5ex, topsep=0pt, partopsep=0pt, itemsep=2pt, parsep=0pt]
\vspace{4pt}\item {No override (telescope model)}
\item {Data file}
\item {Range}
\end{itemize}
}
&
No
\\
\hline

noise/values/components/area/file
&
{Data file containing effective area value(s).}
&
Path name
&
None
\\
\hline

noise/values/components/area/start
&
{Effective area range start value, in m\textsuperscript{2}.}
&
Double
&
0.0
\\
\hline

noise/values/components/area/end
&
{Effective area range end value, in m\textsuperscript{2}.}
&
Double
&
0.0
\\
\hline

noise/values/components/efficiency
&
{Station system efficiency specification type: 
{\begin{itemize}[leftmargin=5ex, topsep=0pt, partopsep=0pt, itemsep=4pt, parsep=0pt]
\vspace{8pt}
 \item {\textbf{No override}: values are loaded from system temperature files found in the telescope model directory.}
 \item {\textbf{Data file}: values are loaded from the specified file.}
 \item {\textbf{Range}: values are evaluated according to the specified range parameters.}
 \vspace{8pt}
\end{itemize}}
}
&
{One of the following:}
{\begin{itemize}[leftmargin=5ex, topsep=0pt, partopsep=0pt, itemsep=2pt, parsep=0pt]
\vspace{4pt}\item {No override (telescope model)}
\item {Data file}
\item {Range}
\end{itemize}
}
&
No
\\
\hline

noise/values/components/efficiency/file
&
{Data file containing system efficiency value(s).}
&
Path name
&
None
\\
\hline

noise/values/components/efficiency/start
&
{System efficiency range start value (allowed range: 0.0 to 1.0).}
&
Double in range 0 $\leq$ $x$ $\leq$ 1
&
0.0
\\
\hline

noise/values/components/efficiency/end
&
{System efficiency range end value (allowed range: 0.0 to 1.0).}
&
Double in range 0 $\leq$ $x$ $\leq$ 1
&
0.0
\\
\hline

oskar\_vis\_filename
&
{Path of the OSKAR visibility output file containing the results of the simulation. Leave blank if not required.}
&
Path name
&

\\
\hline

ms\_filename
&
{Path of the Measurement Set containing the results of the simulation. Leave blank if not required.}
&
Path name
&

\\
\hline

force\_polarised\_ms
&
{If \textbf{True}, always write the Measurment Set in polarised format even if the simulation was run in the single polarisation `Scalar' (or Stokes-I) mode. If \textbf{False}, the size of the polarisation dimension in the the Measurement Set will be determined by the simulation mode.}
&
Bool
&
false
\\
\hline

\end{longtable}
\end{center}
\normalsize
\newpage
