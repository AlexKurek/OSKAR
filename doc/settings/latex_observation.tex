%
% This is an autogenerated file - do not edit!
%


\fontsize{8}{10}\selectfont
\begin{center}
\begin{longtable}{|L{9cm}|L{8.5cm}|L{4.0cm}|L{1.7cm}|}

\hline
  \rowcolor{lightgray}
  {\textbf{Key}} &
  {\textbf{Description}} &
  {\textbf{Allowed values}} &
  {\textbf{Default}} \\ \hline
\endfirsthead
\hline
  \rowcolor{lightgray}
  {\textbf{Key}} &
  {\textbf{Description}} &
  {\textbf{Allowed values}} &
  {\textbf{Default}} \\ \hline
\endhead
  \multicolumn{4}{l}{{Continued on next page\ldots}} \\
\endfoot
 
\endlastfoot
phase\_centre\_ra\_deg
&
{Right Ascension of the observation pointing (phase centre), in degrees.}
&
Double, or CSV list of doubles.
&
0
\\
\hline

phase\_centre\_dec\_deg
&
{Declination of the observation pointing (phase centre), in degrees.}
&
Double, or CSV list of doubles.
&
0
\\
\hline

pointing\_file
&
{Pathname to optional station pointing file, which can be used to override the beam direction for any or all stations in the telescope model. See the accompanying documentation for a description of a station pointing file.}
&
Path name
&
0
\\
\hline

start\_frequency\_hz\textsuperscript{\textbf{\dag}}
&
{The frequency at the midpoint of the first channel, in Hz.}
&
Unsigned double
&

\\
\hline

num\_channels
&
{Number of frequency channels / bands to use.}
&
Integer $\geq$ 1
&
1
\\
\hline

frequency\_inc\_hz
&
{The frequency increment between successive channels, in Hz.}
&
Unsigned double
&
0
\\
\hline

start\_time\_utc\textsuperscript{\textbf{\dag}}
&
{A string describing the start time and date for the observation.}
&
{\vspace{-3.5mm}{\begin{flushleft}
Date-time string of format: \\
\vspace{2pt}
\hspace{2ex}\textbf{`d-M-yyyy h:m:s.z'} \\
\vspace{2pt}
where: \\
\vspace{2pt}
{
\begin{tabular}{@{}p{2ex}@{} @{}p{4.8ex} @{}p{1.25ex}@{} l}
~&\textbf{d}    &  & day number (1 to 31) \\[-0.9ex]
~&\textbf{M}    &  & month (1 to 12)   \\[-0.9ex]
~&\textbf{yyyy} &  & year (4 digits)   \\[-0.9ex]
~&\textbf{h}    &  & hours (0 to 23)   \\[-0.9ex]
~&\textbf{m}    &  & minutes (0 to 59) \\[-0.9ex]
~&\textbf{s}    &  & seconds (0 to 59) \\[-0.9ex]
~&\textbf{z}    &  & milliseconds (0 to 999) \\[-0.9ex]
\end{tabular}
}
\end{flushleft}
}}
&

\\
\hline

length\textsuperscript{\textbf{\dag}}
&
{The observation length either in seconds, or in hours, minutes and seconds.}
&
{\vspace{-6.5mm}{\begin{flushleft}
Double (if length in seconds), or
time string of format: \\
\vspace{2pt}
\hspace{2ex}\textbf{`h:m:s.z'} \\
\vspace{2pt}
where:\\
\vspace{2pt}
{
\begin{tabular}{@{}p{2ex}@{} @{}p{4.8ex} @{}p{1.25ex}@{} l}
~&\textbf{h} &  & hours (0 to 23)   \\[-0.9ex]
~&\textbf{m} &  & minutes (0 to 59) \\[-0.9ex]
~&\textbf{s} &  & seconds (0 to 59) \\[-0.9ex]
~&\textbf{z} &  & milliseconds (0 to 999) \\[-0.9ex]
\end{tabular}
}
\end{flushleft}
}}

&

\\
\hline

num\_time\_steps
&
{Number of time steps in the output data during the observation length. This corresponds to the number of correlator dumps for interferometer simulations, and the number of beam pattern snapshots for beam pattern simulations.}
&
Integer $\geq$ 1
&
1
\\
\hline

\end{longtable}
\end{center}
\normalsize
\newpage
