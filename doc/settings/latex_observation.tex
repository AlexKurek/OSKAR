%
% This is an autogenerated file - do not edit!
%


\fontsize{8}{10}\selectfont
\begin{center}
\begin{longtable}{|c|L{9cm}|L{8.5cm}|L{3.0cm}|L{1.7cm}|}

\hline
  \rowcolor{lightgray}
  {\textbf{ID}} &
  {\textbf{Key}} &
  {\textbf{Description}} &
  {\textbf{Allowed values}} &
  {\textbf{Default}} \\ \hline
\endfirsthead
\hline
  \rowcolor{lightgray}
  {\textbf{ID}} &
  {\textbf{Key}} &
  {\textbf{Description}} &
  {\textbf{Allowed values}} &
  {\textbf{Default}} \\ \hline
\endhead
  \multicolumn{5}{l}{{Continued on next page\ldots}} \\
\endfoot
 
\endlastfoot
093
&
phase\_centre\_ra\_deg
&
Right Ascension of the observation pointing (phase centre), in degrees.
&
Allowed values
&
0
\\
\hline

094
&
phase\_centre\_dec\_deg
&
Declination of the observation pointing (phase centre), in degrees.
&
Allowed values
&
0
\\
\hline

095
&
pointing\_file
&
Pathname to optional station pointing file, which can be used to override the beam direction for any or all stations in the telescope model. See the accompanying documentation for a description of a station pointing file.
&
Path name
&
0
\\
\hline

096
&
start\_frequency\_hz\textsuperscript{\textbf{\dag}}
&
The frequency at the midpoint of the first channel, in Hz.
&
Unsigned double
&

\\
\hline

097
&
num\_channels
&
Number of frequency channels / bands to use.
&
Allowed values
&
1
\\
\hline

098
&
frequency\_inc\_hz
&
The frequency increment between successive channels, in Hz.
&
Unsigned double
&
0
\\
\hline

099
&
start\_time\_utc\textsuperscript{\textbf{\dag}}
&
A string describing the start time and date for the observation.
&
Allowed values
&

\\
\hline

100
&
length\textsuperscript{\textbf{\dag}}
&
The observation length either in seconds, or in hours, minutes and seconds.
&
Allowed values
&

\\
\hline

101
&
num\_time\_steps
&
Number of time steps in the output data during the observation length. This corresponds to the number of correlator dumps for interferometer simulations, and the number of beam pattern snapshots for beam pattern simulations.
&
Allowed values
&
1
\\
\hline

\end{longtable}
\end{center}
\normalsize
\newpage
