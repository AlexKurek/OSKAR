%
% This is an autogenerated file - do not edit!
%


\fontsize{8}{10}\selectfont
\begin{center}
\begin{longtable}{|L{9cm}|L{8.5cm}|L{4.0cm}|L{1.7cm}|}

\hline
  \rowcolor{lightgray}
  {\textbf{Key}} &
  {\textbf{Description}} &
  {\textbf{Allowed values}} &
  {\textbf{Default}} \\ \hline
\endfirsthead
\hline
  \rowcolor{lightgray}
  {\textbf{Key}} &
  {\textbf{Description}} &
  {\textbf{Allowed values}} &
  {\textbf{Default}} \\ \hline
\endhead
  \multicolumn{4}{l}{{Continued on next page\ldots}} \\
\endfoot
 
\endlastfoot
phase\_centre\_ra\_deg
&
{Right Ascension of the observation pointing (phase centre), in degrees.}
&
Double, or CSV list of doubles.
&
0
\\
\hline

phase\_centre\_dec\_deg
&
{Declination of the observation pointing (phase centre), in degrees.}
&
Double, or CSV list of doubles.
&
0
\\
\hline

pointing\_file
&
{Pathname to optional station pointing file, which can be used to override the beam direction for any or all stations in the telescope model. See the accompanying documentation for a description of a station pointing file.}
&
Path name
&

\\
\hline

start\_frequency\_hz\textsuperscript{\textbf{\dag}}
&
{The frequency at the midpoint of the first channel, in Hz.}
&
Unsigned double
&

\\
\hline

num\_channels
&
{Number of frequency channels / bands to use.}
&
Integer > 0
&
1
\\
\hline

frequency\_inc\_hz
&
{The frequency increment between successive channels, in Hz.}
&
Unsigned double
&
0
\\
\hline

start\_time\_utc\textsuperscript{\textbf{\dag}}
&
{The start time and date for the observation. This can be either a MJD value or a string with one of the following formats: 
{\begin{itemize}[leftmargin=5ex, topsep=0pt, partopsep=0pt, itemsep=4pt, parsep=0pt]
\vspace{8pt}
 \item {\textbf{{\texttt{d-M-yyyy h:m:s.z}}}}
 \item {\textbf{{\texttt{yyyy/M/d/h:m:s.z}}}}
 \item {\textbf{{\texttt{yyyy-M-d h:m:s.z}}}}
 \item {\textbf{{\texttt{yyyy-M-dTh:m:s.z}}}}
 \vspace{8pt}
\end{itemize}}
 where: 
{\begin{itemize}[leftmargin=5ex, topsep=0pt, partopsep=0pt, itemsep=4pt, parsep=0pt]
\vspace{8pt}
 \item {\textbf{d} is the day number (1 to 31)}
 \item {\textbf{M} is the month (1 to 12)}
 \item {\textbf{yyyy} is the year (4 digits)}
 \item {\textbf{h} is the hour (0 to 23)}
 \item {\textbf{m} is minutes (0 to 59)}
 \item {\textbf{s} is seconds (0 to 59)}
 \item {\textbf{z} is milliseconds (0 to 999)}
 \vspace{8pt}
\end{itemize}}
}
&
Double (if MJD), or formatted date-time string.
&

\\
\hline

length\textsuperscript{\textbf{\dag}}
&
{The observation length either in seconds, or in hours, minutes and seconds as a formatted string with the syntax \textbf{{\texttt{h:m:s.z}}}, where: 
{\begin{itemize}[leftmargin=5ex, topsep=0pt, partopsep=0pt, itemsep=4pt, parsep=0pt]
\vspace{8pt}
 \item {\textbf{h} is the hour (0 to 23)}
 \item {\textbf{m} is minutes (0 to 59)}
 \item {\textbf{s} is seconds (0 to 59)}
 \item {\textbf{z} is milliseconds (0 to 999)}
 \vspace{8pt}
\end{itemize}}
 Note that values support optional leading zeros in the format string.}
&
Double (if length in seconds), or formatted time string.
&

\\
\hline

num\_time\_steps
&
{Number of time steps in the output data during the observation length. This corresponds to the number of correlator dumps for interferometer simulations, and the number of beam pattern snapshots for beam pattern simulations.}
&
Integer > 0
&
1
\\
\hline

\end{longtable}
\end{center}
\normalsize
\newpage
