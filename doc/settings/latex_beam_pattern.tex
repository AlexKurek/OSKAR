%
% This is an autogenerated file - do not edit!
%


\fontsize{8}{10}\selectfont
\begin{center}
\begin{longtable}{|L{9cm}|L{8.5cm}|L{4.0cm}|L{1.7cm}|}

\hline
  \rowcolor{lightgray}
  {\textbf{Key}} &
  {\textbf{Description}} &
  {\textbf{Allowed values}} &
  {\textbf{Default}} \\ \hline
\endfirsthead
\hline
  \rowcolor{lightgray}
  {\textbf{Key}} &
  {\textbf{Description}} &
  {\textbf{Allowed values}} &
  {\textbf{Default}} \\ \hline
\endhead
  \multicolumn{4}{l}{{Continued on next page\ldots}} \\
\endfoot
 
\endlastfoot
all\_stations
&
{If set, produce beams for all stations in the telescope model; otherwise, for selected stations.}
&
Bool
&
false
\\
\hline

station\_ids
&
{The zero-based station ID number(s) to select from the telescope model when generating beam pattern(s). More than one station ID is specified using a CSV list.}
&
CSV integer list
&
0
\\
\hline

coordinate\_frame
&
{Specification of the coordinate frame in which to evaluate the beam pattern. Horizon-based beam patterns will cover the entire sky.}
&
{One of the following:}
{\begin{itemize}[leftmargin=5ex, topsep=0pt, partopsep=0pt, itemsep=2pt, parsep=0pt]
\vspace{4pt}\item {Equatorial}
\item {Horizon}
\end{itemize}
}
&
Equatorial
\\
\hline

coordinate\_type
&
{Specification of coordinates at which to evaluate the beam pattern. 
{\begin{itemize}[leftmargin=5ex, topsep=0pt, partopsep=0pt, itemsep=4pt, parsep=0pt]
\vspace{8pt}
 \item {\textbf{Beam image:} Tangent plane image, centred on the phase centre direction.}
 \item {\textbf{Sky model:} Evaluate beam only at the supplied coordinates.}
 \vspace{8pt}
\end{itemize}}
}
&
{One of the following:}
{\begin{itemize}[leftmargin=5ex, topsep=0pt, partopsep=0pt, itemsep=2pt, parsep=0pt]
\vspace{4pt}\item {Beam image}
\item {Sky model}
\end{itemize}
}
&
Beam
\\
\hline

beam\_image/specify\_cellsize
&
{If set, specify cellsize; otherwise, specify field of view.}
&
Bool
&
false
\\
\hline

beam\_image/size
&
{Image dimensions. If a single value is specified, the image is assumed to have the same number of pixels along each dimension.
\vspace{8pt}\par\noindent Example: 
{\begin{itemize}[leftmargin=5ex, topsep=0pt, partopsep=0pt, itemsep=4pt, parsep=0pt]
\vspace{8pt}
 \item {A value of `256' results in a square image of size 256 by 256 pixels.}
 \item {A value of `256,128' results in an image of 256 by 128 pixels, with 256 pixels along the Right Ascension direction.}
 \vspace{8pt}
\end{itemize}}
}
&
CSV integer list
&
256
\\
\hline

beam\_image/fov\_deg
&
{Field of view (FOV) in degrees (max 180.0). If a single value is specified, the image is assumed to have the same FOV along each dimension.
\vspace{8pt}\par\noindent Example: 
{\begin{itemize}[leftmargin=5ex, topsep=0pt, partopsep=0pt, itemsep=4pt, parsep=0pt]
\vspace{8pt}
 \item {A value of `2.0' results in an image with a FOV of 2.0 degrees in each dimension.}
 \item {A value of `2.0,1.0' results in an image with a FOV of 2.0 degrees in Right Ascension, and 1.0 degrees in Declination.}
 \vspace{8pt}
\end{itemize}}
}
&
Double, or CSV list of doubles.
&
2.0
\\
\hline

beam\_image/cellsize\_arcsec
&
{The cell (pixel) size in arcseconds.}
&
Unsigned double
&
1.0
\\
\hline

sky\_model/file
&
{Path to an input sky model file.}
&
Path name
&
None
\\
\hline

root\_path
&
{Root path name of the generated data file. Appropriate suffixes and extensions will be added to this, based on the settings below.}
&
Path name
&
None
\\
\hline

output/separate\_time\_and\_channel
&
{Output files without performing any averaging over the time or channel dimensions.}
&
Bool
&
true
\\
\hline

output/average\_time\_and\_channel
&
{Output files after averaging over both the time and channel dimensions.}
&
Bool
&
false
\\
\hline

output/average\_single\_axis
&
{Output files after averaging over the selected dimension.}
&
{One of the following:}
{\begin{itemize}[leftmargin=5ex, topsep=0pt, partopsep=0pt, itemsep=2pt, parsep=0pt]
\vspace{4pt}\item {None}
\item {Time}
\item {Channel}
\end{itemize}
}
&
None
\\
\hline

station\_outputs/text\_file/raw\_complex
&
{If true, save the raw complex pattern in text files.}
&
Bool
&
false
\\
\hline

station\_outputs/text\_file/amp
&
{If true, save each amplitude (voltage) pattern in text files. This is given by the square root of the sum of the squares of the real and imaginary values.}
&
Bool
&
false
\\
\hline

station\_outputs/text\_file/phase
&
{If true, save each phase pattern in text files.}
&
Bool
&
false
\\
\hline

station\_outputs/text\_file/auto\_power
&
{If true, save each total intensity (auto-correlation) beam in text files.}
&
Bool
&
false
\\
\hline

station\_outputs/fits\_image/amp
&
{If true, save each amplitude (voltage) pattern in FITS image files. This is given by the square root of the sum of the squares of the real and imaginary values.}
&
Bool
&
false
\\
\hline

station\_outputs/fits\_image/phase
&
{If true, save each phase pattern in FITS image files.}
&
Bool
&
false
\\
\hline

station\_outputs/fits\_image/auto\_power
&
{If true, save each total intensity (auto-correlation) beam in FITS image files.}
&
Bool
&
false
\\
\hline

station\_outputs/fits\_image/auto\_power\_phase
&
{If true, save the phase of each total intensity (auto-correlation) beam in FITS image files.}
&
Bool
&
false
\\
\hline

station\_outputs/fits\_image/auto\_power\_real
&
{If true, save the real part of each total intensity (auto-correlation) beam in FITS image files.}
&
Bool
&
false
\\
\hline

station\_outputs/fits\_image/auto\_power\_imag
&
{If true, save the imaginary part of each total intensity (auto-correlation) beam in FITS image files.}
&
Bool
&
false
\\
\hline

telescope\_outputs/text\_file/cross\_power\_raw\_complex
&
{If true, save the average cross-power beam raw response from all specified stations as a text file.}
&
Bool
&
false
\\
\hline

telescope\_outputs/text\_file/cross\_power\_amp
&
{If true, save the average cross-power beam amplitude response from all specified stations as text files.}
&
Bool
&
false
\\
\hline

telescope\_outputs/text\_file/cross\_power\_phase
&
{If true, save the average cross-power beam phase response from all specified stations as text files.}
&
Bool
&
false
\\
\hline

telescope\_outputs/fits\_image/cross\_power\_amp
&
{If true, save the average cross-power beam amplitude response from all specified stations in FITS image files.}
&
Bool
&
false
\\
\hline

telescope\_outputs/fits\_image/cross\_power\_phase
&
{If true, save the average cross-power beam phase response from all specified stations in FITS image files.}
&
Bool
&
false
\\
\hline

telescope\_outputs/fits\_image/cross\_power\_real
&
{If true, save the real part of the average cross-power beam response from all specified stations in FITS image files.}
&
Bool
&
false
\\
\hline

telescope\_outputs/fits\_image/cross\_power\_imag
&
{If true, save the imaginary part of the average cross-power beam response from all specified stations in FITS image files.}
&
Bool
&
false
\\
\hline

test\_source/stokes\_i
&
{Use a Stokes I test source.}
&
Bool
&
true
\\
\hline

test\_source/custom
&
{Use a custom test source.}
&
Bool
&
false
\\
\hline

test\_source/custom\_stokes\_i
&
{Stokes I value for the test source.}
&
Unsigned double
&
1.0
\\
\hline

test\_source/custom\_stokes\_q
&
{Stokes Q value for the test source.}
&
Double
&
0.0
\\
\hline

test\_source/custom\_stokes\_u
&
{Stokes U value for the test source.}
&
Double
&
0.0
\\
\hline

test\_source/custom\_stokes\_v
&
{Stokes V value for the test source.}
&
Double
&
0.0
\\
\hline

\end{longtable}
\end{center}
\normalsize
\newpage
