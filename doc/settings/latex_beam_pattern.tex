%
% This is an autogenerated file - do not edit!
%


\fontsize{8}{10}\selectfont
\begin{center}
\begin{longtable}{|L{9cm}|L{8.5cm}|L{4.0cm}|L{1.7cm}|}

\hline
  \rowcolor{lightgray}
  {\textbf{Key}} &
  {\textbf{Description}} &
  {\textbf{Allowed values}} &
  {\textbf{Default}} \\ \hline
\endfirsthead
\hline
  \rowcolor{lightgray}
  {\textbf{Key}} &
  {\textbf{Description}} &
  {\textbf{Allowed values}} &
  {\textbf{Default}} \\ \hline
\endhead
  \multicolumn{4}{l}{{Continued on next page\ldots}} \\
\endfoot
 
\endlastfoot
station\_id
&
{The station ID number (zero based) to select from the telescope model when generating the beam pattern.}
&
Unsigned integer
&
0
\\
\hline

beam\_image/size
&
{Image dimensions. If a single value is specified, the image is assumed to have the same number of pixels along each dimension.
\vspace{8pt}\par\noindent Example: 
{\begin{itemize}[leftmargin=5ex, topsep=0pt, partopsep=0pt, itemsep=4pt, parsep=0pt]
\vspace{8pt} \item {A value of `256' results in a square image of size 256 by 256 pixels.}
 \item {A value of `256,128' results in an image of 256 by 128 pixels, with 256 pixels along the Right Ascension direction.}
 \vspace{8pt}
\end{itemize}}
}
&
CSV integer list
&
256
\\
\hline

beam\_image/fov\_deg
&
{Field-of-view (FOV) in degrees (max 180.0). If a single value is specified, the image is assumed to have the same FOV along each dimension. 
\vspace{8pt}\par\noindent Example: 
{\begin{itemize}[leftmargin=5ex, topsep=0pt, partopsep=0pt, itemsep=4pt, parsep=0pt]
\vspace{8pt} \item {A value of `2.0' results in an image with a FOV of 2.0 degrees in each dimension.}
 \item {A value of `2.0,1.0' results in an image with a FOV of 2.0 degrees in Right Ascension, and 1.0 degrees in Declination.}
 \vspace{8pt}
\end{itemize}}
}
&
Double, or CSV list of doubles.
&
2.0
\\
\hline

root\_path
&
{Root path name of the generated data file. Appropriate suffixes and extensions will be added to this, based on the settings below.}
&
Path name
&
None
\\
\hline

oskar\_image\_file/save\_voltage
&
{If true, save the voltage amplitude pattern in an OSKAR image file. (This is given by the square root of the sum of the squares of the real and imaginary values.)}
&
Bool
&
false
\\
\hline

oskar\_image\_file/save\_phase
&
{If true, save the phase pattern in an OSKAR image file.}
&
Bool
&
false
\\
\hline

oskar\_image\_file/save\_complex
&
{If true, save the complex (real and imaginary) pattern in an OSKAR image file.}
&
Bool
&
false
\\
\hline

oskar\_image\_file/save\_total\_intensity
&
{If true, save the total intensity beam. This is evaluated as the Stokes I response of the beam pattern auto-correlation.}
&
Bool
&
false
\\
\hline

fits\_file/save\_voltage
&
{If true, save the voltage amplitude pattern in a FITS image file. (This is given by the square root of the sum of the squares of the real and imaginary values.)}
&
Bool
&
false
\\
\hline

fits\_file/save\_phase
&
{If true, save the phase pattern in a FITS image file.}
&
Bool
&
false
\\
\hline

fits\_file/save\_total\_intensity
&
{If true, save the total intensity beam. This is evaluated as the Stokes-I response of the beam pattern auto-correlation.}
&
Bool
&
false
\\
\hline

\end{longtable}
\end{center}
\normalsize
\newpage
